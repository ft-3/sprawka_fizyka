\documentclass[11pt]{article}
\usepackage{caption}

\title{\textbf{Opracowanie danych pomiarowych}}
\author{Tymoteusz Chmielecki\\
		Mateusz Bałuch}
\date{02.03.2020}
\begin{document}

\maketitle

\section{Cel ćwiczenia}
Celem ćwiczenia było zapoznanie sie z metodami opracowywania i wykorzystywania wyników pomiarowych, w tym celu użyte zostało wahadło proste.

\section{Wstep teoretyczny}
\subsection{Niepewność pomiarowa}
Niepewność pomiaru to parametr zwiazany z wynikiem pomiaru, charakteryzujacy rozrzut wyników, które można w uzasadniony sposób przypisać wartości mierzonej. Charakteryzuje ona
rozrzut wartości (szerokość przedziału), wewnatrz którego można z zadowalajacym prawdopodobieństwem usytuować wartość wielkości mierzonej. Z definicji niepewności pomiarowej wynika, że nie może być ona wyznaczona doskonale dokładnie.
\subsection{Wahadło matematyczne}
Wahadłem matematycznym jest punktowa masa zawieszona na nieważkiej nici. Na potrzeby ćwiczenia użyliśmy kuli zawieszonej na cienkiej nici. Wychylamy wahadło z położenia równowagi i wprowadzamy je w ruch drgajacy prosty. Dana zależność opisuje okres jego drgań: $$ T = 2\pi\sqrt{\frac{l}{g}} $$
Po przekształceniu otrzymujemy wzór na przyśpieszenie ziemskie jako funkcji okresu i długości nici: $$ g = \frac{4{\pi}^2l}{T^2} $$

\section{Układ pomiarowy}
Do przeprowadzenia pomiarów użyliśmy wahadła matematycznego złożonego z obciażnika zawieszonego na cienkiej nici przyczepionej do statywu. Jako przyrzadow pomiarowych użyliśmy stopera oraz przymiaru milimetrowego.

\section{Wykonanie ćwiczenia}
Na ćwiczenie złożyły sie 2 cześci.
\subsection{Wyznaczenie $g$ na podstawie pomiaru 6 serii 20 wahań}
\subsection{Badanie zależności $T$ od $l$}

\section{Wyniki}
\begin{minipage}{\linewidth}
\centering
\captionof{table}{Pomiary okresu drgań przy ustalonym $l = 0.404m$} \label{tab:title} 
\begin{tabular}{c | c | c | c}
	Pomiar & Liczba okresów $k$ & Czas $t$ dla $k$ okresów $[s]$ & Okres $T_i = ^t/_k [s]$\\ \hline
	1 & 20 & 25.00 & 1.25\\
	2 & 20 & 25.16 & 1.258\\
	3 & 20 & 25.22 & 1.261\\
	4 & 20 & 25.28 & 1.264\\
	5 & 20 & 25.31 & 1.2655\\
	6 & 20 & 25.37 & 2.2685\\
\end{tabular}\par
\end{minipage}


\end{document}
