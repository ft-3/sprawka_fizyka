\documentclass[11pt]{article}
\usepackage{caption}
\usepackage{amsmath}
\usepackage[T1]{fontenc}
\usepackage{tikz}

\title{\textbf{Opracowanie danych pomiarowych}}
\author{Tymoteusz Chmielecki\\
		Mateusz Bałuch}
\date{02.03.2020}
\begin{document}

\maketitle

\section{Cel ćwiczenia}
Celem ćwiczenia było zapoznanie sie z metodami opracowywania i wykorzystywania wyników pomiarowych, w tym celu użyte zostało wahadło proste.

\section{Wstep teoretyczny}
\subsection{Niepewność pomiarowa}
Niepewność pomiaru to parametr zwiazany z wynikiem pomiaru, charakteryzujacy rozrzut wyników, które można w uzasadniony sposób przypisać wartości mierzonej. Charakteryzuje ona
rozrzut wartości (szerokość przedziału), wewnatrz którego można z zadowalajacym prawdopodobieństwem usytuować wartość wielkości mierzonej. Z definicji niepewności pomiarowej wynika, że nie może być ona wyznaczona doskonale dokładnie.
\subsection{Wahadło matematyczne}
Wahadłem matematycznym jest punktowa masa zawieszona na nieważkiej nici. Na potrzeby ćwiczenia użyliśmy kuli zawieszonej na cienkiej nici. Wychylamy wahadło z położenia równowagi i wprowadzamy je w ruch drgajacy prosty. Dana zależność opisuje okres jego drgań: $$ T = 2\pi\sqrt{\frac{l}{g}} \refstepcounter{equation}\eqno(\theequation)$$
Po przekształceniu otrzymujemy wzór na przyśpieszenie ziemskie jako funkcji okresu i długości nici: $$ g = \frac{4{\pi}^2l}{T^2} \refstepcounter{equation}\eqno(\theequation)$$

\section{Układ pomiarowy}
Do przeprowadzenia pomiarów użyliśmy wahadła matematycznego złożonego z obciażnika zawieszonego na cienkiej nici przyczepionej do statywu. Jako przyrzadow pomiarowych użyliśmy stopera oraz przymiaru milimetrowego.

\section{Wykonanie ćwiczenia}
Na ćwiczenie złożyły sie 2 cześci.
\subsection{Wyznaczenie $g$ na podstawie serii pomiarów wahań}
Wprawiliśmy wahadło w ruch a następnie korzystając ze stopera zmierzyliśmy czas t potrzebny na wykonanie 20 drgań. 
Zmierzyliśmy długość nici na której zawieszony był obciążnik przy użyciu przymiaru milimetrowego.

\subsection{Badanie zależności $T$ od $l$}
Podobnie jak w poprzednim punkcie wykonano pomiar czasu t potrzebnego na wykonanie 20 okresów drgania wahadła. Długość nici stopniowo zmniejszaliśmy od 51.5cm do 12.5cm.

\clearpage
\section{Wyniki}
\begin{minipage}{\linewidth}
\centering
\captionof{table}{Pomiary okresu drgań przy ustalonym $l = 0.404m$} \label{tab:title} 
\begin{tabular}{c | c | c | c}
	Pomiar & Liczba okresów $k$ & Czas $t$ dla $k$ okresów $[s]$ & Okres $T_i = \frac{t}{k} [s]$\\ \hline
	1 & 20 & 25.00 & 1.25\\
	2 & 20 & 25.16 & 1.258\\
	3 & 20 & 25.22 & 1.261\\
	4 & 20 & 25.28 & 1.264\\
	5 & 20 & 25.31 & 1.2655\\
	6 & 20 & 25.37 & 1.2685\\
\end{tabular}\par
\end{minipage}
\\
\\
\begin{minipage}{\linewidth}
\centering
\captionof{table}{Pomiar zależności okresu drgań od długości wahadła} \label{tab:title} 
\begin{tabular}{c | c | c | c | c | c}
	Pomiar & $L [cm]$ & $k$  & $t [s]$ & $T_i [s] $ & $g_i [\frac{m}{s^2}]$\\ \hline
	1 & 51.5 & 20 & 28.56 & 1.428 & 9.97\\
	2 & 44.2 & 20 & 26.56 & 1.328 & 9.89\\
	3 & 36.5 & 20 & 24.16 & 1.208 & 9.87\\
	4 & 29.3 & 20 & 21.47 & 1.0735 & 10.04\\
	5 & 20.8 & 20 & 18.12 & 0.906 & 10.00\\
	6 & 12.5 & 20 & 13.97 & 0.6985 & 10.11\\
\end{tabular}\par
\end{minipage}

\clearpage
\section{Opracowanie wyników}
Do obliczeń przyjmujemy, że refleks człowieka wynosi około 0.3 [s]. Wynika z tego, że dokładność pomiaru okresu przy uwzglednieniu dokładności stopera wynosi: $$ u(T) = \sqrt{(0.3s)^2+(0.01s)^2} \approx 0.3s $$
Wyznaczamy również niepewność pomiaru długości wahadła. Zmierzona została ona przymiarem milimetrowym otrzymujac wartość $l = 0.404 m$. Przy niepewności skali $u(l) = 0.001 m$ oraz trudności przyłożenia miarki do środka kuli, końcowa niepewność długości oszacowaliśmy na $$u(l) = 0.002 m$$.
\subsection{Wyznaczenie $g$ na podstawie serii pomiarów wahań}
Zakładamy, że wyniki nie zawieraja błedów grubych z uwagi na różnice miedzy najwieksza i najmniejsza wartościa obliczonego przez nas $T_i$.\\
Wartościa okresu, która przyjmujemy jest średnia arytmetyczna z uzyskanych pomiarów: $$ T_{sr} = \frac{1}{n}\displaystyle\sum_{i=1}^n T_i \refstepcounter{equation}\eqno(\theequation) $$ Dla uzyskanych przez nas danych eksperymentalnych $ T_{sr} = 1.261 $
Przyśpieszenie ziemskie $g$ wyznaczone ze wzoru $(2)$ wynosi: $$ g = 10.02757 \frac{m}{s^2} \approx 10.03 \frac{m}{s^2} $$
Wynaczyliśmy również niepewność (typu A) pomiaru okresu ze wzoru: $$ u(T_{sr}) = \sqrt{\frac{1}{n(n-1)}\displaystyle\sum_{i=1}^n (T_i - T_{sr})} \refstepcounter{equation}\eqno(\theequation) $$
Po podstawieniu danych eksperymentalnych otrzymaliśmy $u(T_{sr}) \approx 0 s$
Nastepnie wyznaczyliśmy niepewność złożona ze wzoru:
$$ u_c(g) = \sqrt{(\frac{{\partial}g}{{\partial}T})^2 \cdot u(T)^2 + (\frac{{\partial}g}{{\partial}l})^2 \cdot u(l)^2} \refstepcounter{equation}\eqno(\theequation) $$
co dało nam wynik: $$ u_c(g) = \sqrt{(\frac{64\pi^4l^2}{T^6}) \cdot u(T)^2 + (\frac{16\pi^4}{T^4}) \cdot u(l)^2} \approx 0.122 \ \frac{m}{s^2}$$
Aby porównać otrzymana wartość z tablicowa obliczymy niepewność rozszerzona ze wzoru: $$ U_c(g) = k \cdot u_c(g) = 2 \cdot 0.122\ \frac{m}{s^2} = 0.244\ \frac{m}{s^2} \approx 0.24\ \frac{m}{s^2} $$
Czyli przyspieszenie ziemskie ma wartość: $$ g = (10.03 \pm 0.24)\ \frac{m}{s^2}$$
Wartość tablicowa ($g = 9.80665\ \frac{m}{s^2}$) przyśpieszenie ziemskiego mieści sie w wyznaczonym przedziale co oznacza, że pomiary zostały poprawnie wykonane.

\subsection{Badanie zależności $T$ od $l$}
Podnoszac do kwadratu wzór $(1)$ otrzymujemy nastepujaca zależność: $$ T^2 = \frac{4\pi^2}{g} \cdot l \refstepcounter{equation}\eqno(\theequation) $$
Otrzymujemy wzór funkcji $T^2$ w zależności od l. Jest to funkcja liniowa o współczynniku kierunkowym: $$ a = \frac{4\pi^2}{g} \refstepcounter{equation}\eqno(\theequation) $$

Czynniki otrzymane metodą regresji liniowej wynoszą: $$a = 4.00005\frac{m}{s^2} \approx 4\frac{m}{s^2}$$  $$b=-0.0120153\frac{m}{s^2} \approx -0.012\frac{m}{s^2}$$
$$U(a)=0.109\frac{m}{s^2} $$

Na podstawie niepewności pomiarowej $U(a)$ obliczona została niepewność $U(g)$ zgodnie ze wzorem:
$$ U(g) = \sqrt{(\frac{4\pi^2}{a^2})^2 \cdot u(a)^2} = \frac{4\pi^2}{a^2} \cdot u(a) $$
Po podstawieniu danych:
$$ U(g) = \frac{4\pi^2}{4^2} \cdot 0.109 = 0.085565 \approx 0.085 $$

Otrzymaliśmy wartość przyśpieszenia ziemskiego równą:
$$ g = \frac{4\pi^2}{a} \approx (9.86 \pm 0.085)\ \frac{m}{s^2} $$

Wartość tablicowa ($g = 9.80665\ \frac{m}{s^2}$) przyśpieszenie ziemskiego mieści sie w wyznaczonym przedziale co oznacza, że pomiary zostały poprawnie wykonane.



\clearpage
\section{Wnioski}
\begin{enumerate}
	\item Dokonane pomiary pozwoliły obliczyć wartość przyśpieszenia ziemskiego która wyniosła $ g = (10.03 \pm 0.24)\ \frac{m}{s^2}$
	\item Metoda regresji liniowej zależności $T^2$ od L pozwoliła wyliczyć wartość przyśpieszenia ziemskiego równą $ g = (9.86 \pm 0.085)\ \frac{m}{s^2}$
	\item Dokładność wyników zależy od refleksu osoby odpowiedzialnej za pomiar czasu oraz od odpowiedniego doobrania wychylenia by kąt był odpowiednio mały
	\item Wyznaczanie przyśpieszenia grawitacyjnego Ziemi przy pomocy wahadła daje wyniki zbliżone do tablicowych
\end{enumerate}


\end{document}
